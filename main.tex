\documentclass{article}
\usepackage[utf8]{inputenc}
\usepackage{graphicx}
\usepackage{hyperref}

\begin{document}
\begin{figure}
    \centering
    \includegraphics[scale = 0.25]{images/gmit.jpg}
    \label{fig:gmit}
\end{figure}

\title{Pixel Wizard Test Plan}
\author{Prepared by: Arnas Steponavicius }
\date{May 2020}

\maketitle

\tableofcontents
\newpage

\section{Introduction}
    This game will be a 2D side-scrolling platformer, inspired by the likes of ‘\textbf{Salt and Sanctuary}’, ‘\textbf{Shovel  Knight}’, and ‘\textbf{Fancy  Pants}’, with elements from ‘\textbf{Skyrim}’ (mainly in the way the player character and enemy characters attack).
    The artwork will be inspired mainly by Shovel Knight, which uses mainly pixel art to create its characters and world. \newline
    The gameplay will be inspired by ‘\textbf{Salt and Sanctuary}’ and ‘\textbf{Dark Souls}’ and ‘\textbf{Skyrim}’, which will see the player navigate  progressively  difficult  levels  with  a  wizard type  character  that  uses magic a lá ‘\textbf{Skyrim}’.\newline
    Each level will have several enemies that the player must defeat to progress. Each level will also have a boss that the player must defeat to progress to the next level. Each level will contain pickups for the player, such as health pickups to replenish the player’s health.
    \newline
    \newline

    \textbf{Game Characteristics} 
    \newline 
    As this will be a platformer, this game will: 
    \newline
    \textbullet{ Allow the player to control a specific character, that has an important fictional/narrative role.}
    \newline
    \textbullet{ Have game statistics and/or relational attributes with other game objects, enemies, and/or the player character.} 
    \newline
    \textbullet{ Allow  the  player to take on and navigate the levels using an easy-to-use  user interface.}
    \newline
    \textbullet{ Have obstacles that the player must overcome, such as enemies and bosses.}
\newpage

\section{Objectives and Tasks}
    \subsection{Objectives}
    \subsection{Tasks}
\newpage

\section{Scope}
    \subsection{General}
    \subsection{Tactics}
\newpage

\section{Testing Strategy}
    \subsection{Unit Testing}
    \subsection{System and Integration Testing}
    \subsection{Performance and Stress Testing}
    \subsection{User Acceptance Testing}
    \subsection{Batch Testing}
    \subsection{Automated Regression Testing}
    \subsection{Beta Testing}
\newpage

\section{Test Schedule}
\newpage

\section{Control Procedures}
\newpage

\section{Features to be tested}
\newpage

\section{Features NOT to be tested}
\newpage

\section{Resources/Roles \& Responsibilities}
\newpage

\section{Schedules}
\newpage

\section{Risks/Assumptions}
\newpage

\section{Tools}
\newpage

\end{document}
